\documentclass[letterpaper, 12pt]{extarticle}
% The sizes available are 8pt, 9pt, 10pt, 11pt, 12pt, 14pt, 17pt, and 20pt.

\usepackage[scale=0.9]{geometry}

\usepackage{scrextend}

\setlength{\parindent}{0pt}

\setlength{\parskip}{5pt}

\usepackage{booktabs}

\usepackage{enumitem}
\setlist[itemize]{topsep=0pt, parsep=0pt, itemsep=0pt}

\usepackage{authblk}

\usepackage{natbib}
\bibliographystyle{plainnat}

\usepackage[bookmarksnumbered]{hyperref}

\usepackage{enumitem}
\setlist[enumerate]{label*=\arabic*.}

\usepackage[yyyymmdd]{datetime}

\usepackage{amsmath}
\usepackage{amsthm}
\usepackage{amssymb}

\theoremstyle{definition} % non-italic
\newtheorem{theorem}{Theorem}[subsection]
\newtheorem{lemma}[theorem]{Lemma}
\newtheorem{proposition}[theorem]{Proposition}
\newtheorem{corollary}[theorem]{Corollary}
\newtheorem{remark}[theorem]{Remark}
\newtheorem{definition}[theorem]{Definition}
\newtheorem{example}[theorem]{Example}

\usepackage[none]{hyphenat}

\usepackage{fancyhdr}
\pagestyle{fancy}

\fancyhf{}
\renewcommand{\headrulewidth}{0pt}

\usepackage{lastpage}
\rfoot{Page \thepage{} of \pageref{LastPage}}

%%%%%%%%%%%%%%%%%%%%%%%%%%%%%%%%%%%%%%%%%%%%%%%%%%%%%%%%%%%%

%:commands

\newcommand{\textdef}[1]{\textbf{#1}}
\newcommand{\code}[1]{\texttt{#1}}
\newcommand{\kn}{\code{Knotty}}
\newcommand{\kne}{\code{Knotty Engine}}
\newcommand{\tex}{\TeX}

\newcommand{\set}[1]{\{#1\}}
\newcommand{\op}{\diamond}

%%%%%%%%%%%%%%%%%%%%%%%%%%%%%%%%%%%%%%%%%%%%%%%%%%%%%%%%%%%%

% environments

\newenvironment{codeblock}
    {\begin{addmargin}{0.5in} \ttfamily}
    {\end{addmargin}}

%%%%%%%%%%%%%%%%%%%%%%%%%%%%%%%%%%%%%%%%%%%%%%%%%%%%%%%%%%%%

\title{The \kn{} Companion}

\author{Vu Phan}

\begin{document}

\maketitle

\begin{abstract}
This document is the specification of the \kn{} language.
\end{abstract}

\tableofcontents

\thispagestyle{fancy}

%%%%%%%%%%%%%%%%%%%%%%%%%%%%%%%%%%%%%%%%%%%%%%%%%%%%%%%%%%%%

\newpage

\section{Syntax}

%%%%%%%%%%%%%%%%%%%%%%%%%%%%%%%%%%%%%%%%%%%%%%%%%%%%%%%%%%%%

\subsection{Lexicon}

These are \textdef{reserved lexemes}:
\begin{codeblock}
unknown constant function let return check \\
, ( ) := \\
if else \\
true false
\end{codeblock}

These are \textdef{operators}:
\begin{codeblock}
or and not \\
= /= > < >= <= \\
+ - * / \% \^{}
\end{codeblock}

A \textdef{number} is either:
\begin{itemize}
\item \code{i}, or
\item one or more consecutive digits (\code{0}-\code{9})
\end{itemize}

An \textdef{identifier} is one letter
(\code{a}-\code{z} or \code{A}-\code{Z})
followed by zero or more letters and digits.
Also, an identifier must not be a reserved lexeme,
an operator, or a number.

Note: blank characters (spaces, tabs, new-lines)
are delimiters.

%%%%%%%%%%%%%%%%%%%%%%%%%%%%%%%%%%%%%%%%%%%%%%%%%%%%%%%%%%%%

\subsection{Grammar}

\subsubsection{Names}

An \textdef{unknown name} is an identifier.
So is a \textdef{constant name}, a \textdef{function name},
a \textdef{temporary name}, and a \textdef{check name}.

\subsubsection{Named Terms}

A \textdef{named term} is an unknown name, a constant name,
a function name, a formal parameter, a temporary name, or
an actual function expression.

\subsubsection{Parameters}

A \textdef{formal parameter} is an identifier.
An \textdef{actual parameter} is a term.

\subsubsection{Function Expressions}

A \textdef{formal function expression} has the form
\begin{codeblock}
$f$($p_1$,$\ldots$, $p_n$)
\end{codeblock}
where $f$ is a function name and
each $p_i$ is a formal parameter ($n > 0$).

An \textdef{actual function expression} has the same form,
but with each $p_i$ being an actual parameter.

\subsubsection{Arithmetic Terms}

An \textdef{inner arithmetic term}:
\begin{itemize}
\item
is a number, or
\item
is a named term, or
\item
has the form
\begin{codeblock}
($t$)
\end{codeblock}
where $t$ is an inner arithmetic term.
\end{itemize}


An \textdef{arithmetic term}:
\begin{itemize}
\item
is an inner arithmetic term, or
\item
has the form
\begin{codeblock}
-$t$
\end{codeblock}
where $t$ is an arithmetic term, or
\item
has the form
\begin{codeblock}
$t_1 \op t_2$
\end{codeblock}
where $t_1$ \& $t_2$ are terms and
$\op \in \set{+, -, *, /, \%, \hat~}$.
\end{itemize}

\subsubsection{Boolean Terms}

A \textdef{comparison boolean term} has the form
\begin{codeblock}
$t_1 \op t_2$
\end{codeblock}
where $t_1$ \& $t_2$ are arithmetic terms and
$\op \in \set{=, /=, >, <, >=, <=}$.

An \textdef{inner boolean term}:
\begin{itemize}
\item
is the keyword \code{true}, or
\item
is the keyword \code{false}, or
\item
is a named term, or
\item
has the form
\begin{codeblock}
($t$)
\end{codeblock}
where $t$ is an inner boolean term.
\end{itemize}

A \textdef{boolean term}:
\begin{itemize}
\item
is a comparison boolean term, or
\item
is an inner boolean term, or
\item
has the form
\begin{codeblock}
not $t$
\end{codeblock}
where $t$ is a boolean term, or
\item
has the form
\begin{codeblock}
$t_1 \op t_2$
\end{codeblock}
where $t_1$ \& $t_2$ are boolean terms and
$\op \in \set{\code{or}, \code{and}}$.
\end{itemize}

\newpage

\subsubsection{Terms}

An \textdef{inner term}:
\begin{itemize}
\item
is a named term, or
\item
has the form
\begin{codeblock}
($t$)
\end{codeblock}
where $t$ is an inner term.
\end{itemize}

A \textdef{conditional term} has the form
\begin{codeblock}
$t_1$ if $t_2$ else $t_3$
\end{codeblock}
where $t_1$ \& $t_3$ are terms and $t_2$ is a boolean term.

A \textdef{term} is an inner term, an arithmetic term,
a boolean term, or a conditional term.

\subsubsection{Unknown Statements}

An \textdef{unknown statement} has the form
\begin{codeblock}
unknown $u_1$,$\ldots$, $u_n$
\end{codeblock}
where each $u_i$ is an unknown name ($n > 0$).

\subsubsection{Constant Statements}

A \textdef{constant statement} has the form
\begin{codeblock}
constant $c$ := $t$
\end{codeblock}
where $c$ is a constant name and $t$ is a term.

\subsubsection{Function Statements}

A \textdef{return clause} has the form
\begin{codeblock}
return $t$
\end{codeblock}
where $t$ is a term.

A \textdef{let clause} has the form
\begin{codeblock}
let $tmp$ := $t$
\end{codeblock}
where $tmp$ is a temporary name and $t$ is a term.

A \textdef{function statement} has the form
\begin{codeblock}
function $fe$
    \begin{codeblock}
    $lc_1 \\
    \ldots \\
    lc_n$ \\
    $rc$
    \end{codeblock}
\end{codeblock}
where $fe$ is a formal function expression,
each $lc_i$ is a let clause ($n \ge 0$), and
$rc$ is a return clause.

\subsubsection{Check Statements}

A \textdef{check statement} has the form
\begin{codeblock}
check $c$ := $t$
\end{codeblock}
where $c$ is a check name and $t$ is a term.

\subsubsection{Program}

A \textdef{program} has the form
\begin{codeblock}
$s_1 \\
\ldots \\
s_n$
\end{codeblock}
where each $s_i$ is an unknown statement,
a constant statement, a function statement, or
a check statement ($n \ge 0$).

%%%%%%%%%%%%%%%%%%%%%%%%%%%%%%%%%%%%%%%%%%%%%%%%%%%%%%%%%%%%

\begin{codeblock}

\end{codeblock}

\section{Semantics}

\subsection{Operations}

\begin{tabular}{l|l}
Operator & Meaning \\ \toprule
\code{or} & disjunction \\
\code{and} & conjunction \\
\code{not} & negation \\
= & equal \\
/= & unequal \\
\textgreater & greater \\
\textless & less \\
\textgreater= & greater or equal \\
\textless= & less or equal \\
+ & plus \\
- & unary or binary minus \\
* & multiplication \\
/ & division \\
\% & modulo \\
\^{} & exponentiation
\end{tabular}

Note: parentheses override usual operator precedence.

\subsection{Namespace}

\begin{tabular}{l|l}
Name & Scope \\ \toprule
unknown & program \\
constant & program \\
function & program \\
temporary & function
\end{tabular}

\subsection{Values}

The number \code{i} is the imaginary unit.

An unknown name represents a complex number
of unspecified value.

A constant name represents the term
in the corresponding constant statement.

A function name represents the mapping
defined in the corresponding function statement.

A temporary name represents the term
in the corresponding temporary clause.

A check name represents the term
in the corresponding check statement.

\subsection{Input/Output}

The \kne:
\begin{itemize}
\item
accepts a \kn{} program
\item
generates a \tex{} program showing the check names and
their corresponding values
as specified by the \kn{} program.
\end{itemize}

%%%%%%%%%%%%%%%%%%%%%%%%%%%%%%%%%%%%%%%%%%%%%%%%%%%%%%%%%%%%

\end{document}
